\synctex=1
\documentclass[
   paper=a4,
   fontsize=11pt,
   parskip=no,       % FIXME: I want this to be `half`, but that breaks *all* my spacing rn
   fleqn             % Align math left? o_O
]{scrartcl}

\usepackage[svgnames]{xcolor}
\usepackage{graphics}
\usepackage[
   textheight=700px,
   head=18.27997pt
]{geometry}
\usepackage{etoolbox}

\usepackage{scrlayer-scrpage}
\usepackage{enumitem}   % To adjust the spacing of multiline ‘definition lists’
\usepackage{multicol}

\usepackage{mathtools}
\usepackage{amsfonts,amsthm,amssymb}
\usepackage[math-style=TeX]{unicode-math}
   \setmathfont{xits-math.otf}
%  \setmathfont[Scale=MatchLowercase,math-style=upright,vargreek-shape=unicode]{Neo Euler}
%  \setmathfont[range=\mathbfsfit/{greek,Greek,latin,Latin}]{EB Garamond}

\usepackage{fontspec}
   %\defaultfontfeatures{}
   \setmainfont{EB Garamond}[Numbers={OldStyle,Proportional}]
   \setsansfont{Linux Biolinum O}
   \setmonofont{Fantasque Sans Mono}
\setkomafont{disposition}{\normalfont}
\usepackage{bm}         % Bold-math handling

\usepackage{url}        % Pretty, line-breaking URLs                                            \url
\usepackage{currfile}   % Get the current filename (for the footer)                    \currfilename
\usepackage{lastpage}   % A reference to the page-number of the last page         \pageref{LastPage}

\usepackage{blindtext}

%%% === Layout ===
%\pagestyle{empty}
\automark[subsection]{section}
\pagestyle{scrheadings}

%%% Title
\setkomafont{author}{\scshape}

\title{Assignment 2}
\subtitle{CS 330}
\author{elliottcable}

%%% Header / footer
\setkomafont{pagehead}{\normalfont\sffamily\upshape}
\setkomafont{pagefoot}{\normalfont\sffamily\upshape}
\setkomafont{pagenumber}{\normalfont\sffamily\upshape}

\renewcommand\pagemark{{\usekomafont{pagenumber}%
   \thepage\nobreakspace of\nobreakspace\pageref{LastPage}%
}}

\lohead{\bfseries\headmark}    %% Top left on odd pages (and single-sided)
\rohead{\bfseries\pagemark}    %% Top right on odd pages (and single-sided)
\lofoot{\ttfamily\currfilename}
\rofoot{\today}
\chead{}                       %% Top center
\cfoot{}                       %% Bottom center

%%% Headings
%\usepackage{sectsty}
%   \sectionfont{%
%      \usefont{OT1}{phv}{b}{n}%
%      \sectionrule{0pt}{0pt}{-5pt}{3pt} }

%%% Patch the title to remove the date
\makeatletter
\patchcmd{\@maketitle}% <cmd>
   {{\usekomafont{date}{\@date \par}}%
      \vskip \z@ \@plus 1em}% <search>
   {}% <replace>
   {}{}% <success><failure>
\makeatother

%%% Columns
\setlength\columnsep{3.5em}
\setlength{\columnseprule}{0.5pt}
\def\columnseprulecolor{\color{gray}}

%%% Text
\setlength{\parindent}{3ex}
\setlength{\mathindent}{1em}

%%% Macros
% ‘Boolean true’ (and false)
\newcommand{\BT}{$\bm{\mathsf{T}}$}
\newcommand{\BF}{$\bm{\mathsf{F}}$}
\renewcommand{\qed}{\hfill\blacksquare}

% ‘Logical math’ and ‘logical variable’
%\definecolor{light-gray}{gray}{0.89}
%\newcommand{\Pm}[1]{\colorbox{light-gray}{\ensuremath{\mathsf{#1}}}}
%\newcommand{\Pv}[1]{\colorbox{light-gray}{\ensuremath{\mathit{#1}}}}
\newcommand{\Pm}[1]{\ensuremath{\mathsf{#1}}}
\newcommand{\Pv}[1]{\ensuremath{\mathit{#1}}}

%\usepackage{showframe}

%%% === Content ===
\begin{document}
\maketitle

% ==== ==== ====
\section*{1.4 | № 48}
\begin{addmargin}[2.5em]{2.5em}{\sffamily
      “Establish these logical equivalences, where \Pv{x} does not occur as a free variable in
      \Pm{A}. Assume that the domain is nonempty.”
}\end{addmargin}

\begin{multicols}{2}
\begin{enumerate}[
      leftmargin=0pt, labelsep=0.25em,
      label=\textsf{\textbf{\alph*)}}
]
\raggedcolumns

   % The \begin{aligned} mess is to center an equation *and* keep it on the same line as the item.
   \item\hfil$\begin{aligned}[t]
      \Pm{\forall x (A \rightarrow P(x)) \equiv A \rightarrow \forall x P(x)}
   \end{aligned}$

   In both the left- and right-hand sides, there are two basic cases of truth: the first where
   \Pm{A} is \BF, and the second where it is \BT.
   \begin{description}[labelwidth=1em, labelsep=0em, leftmargin=1em]
      \item[F]
      When \Pm{A} is \BF, we find that the implication \Pm{A \rightarrow P(x)} is \textit{always}
      vacuously true, regardless of the value of \Pm{P(x)}.

      Since this reasoning applies to the expressions on both sides of the equivalence, regardless
      of the presence or absence of quantification around \Pm{A}, in this case, the expressions are
      confirmed equivalent.

      \item[T]
      When \Pm{A} is \BT, we can conclude that the truth-value of the left-hand expression relies
      entirely on that of \Pm{P(x)}, quantified by \Pm{\forall x}. As this is exactly what is
      expressed by the right-hand side, we can conclude that in this case, too, the expressions are
      equivalent.
   \end{description}

   As we have shown the two expressions to be equivalent in every case, we can conclude them to be
   logically equivalent. \qed

\columnbreak

   \item\hfil$\begin{aligned}[t]
      \Pm{\exists x (A \rightarrow P(x)) \equiv A \rightarrow \exists x P(x)}
   \end{aligned}$

   As with the previous, there are two cases for both of these expressions: the first, where \Pm{A}
   is \BF, and a second where it is \BT.
   \begin{description}[labelwidth=1em, labelsep=0em, leftmargin=1em]
      \item[F]
      Again, here, the implication \Pm{A \rightarrow P(x)} is vacuously true.

      Also again, as this reasoning applies to the both expressions, regardless
      of the quantification, in this case, the expressions are equivalent.

      \item[T]
      When \Pm{A} is \BT, the truth-value of the left-hand expression relies on that of \Pm{P(x)},
      quantified by \Pm{\exists x}. As this is, again, exactly what is expressed by the right-hand
      side, we conclude that these expressions are equivalent. \qed
   \end{description}

\end{enumerate}
\end{multicols}

% ==== ==== ====
\section*{1.4 | № 62}
\begin{addmargin}[2.5em]{2.5em}{\sffamily
   “Let \Pm{P(x)}, \Pm{Q(x)}, \Pm{R(x)}, and \Pm{S(x)} be the statements ‘\Pv{x} is a duck,’ ‘\Pv{x}
   is one of my poultry,’ ‘\Pv{x} is an officer,’ and ‘\Pv{x} is willing to waltz,’ respectively.
   Express each of these statements using quantifiers; logical connectives; and \Pm{P(x)},
   \Pm{Q(x)}, \Pm{R(x)}, and \Pm{S(x)}.”
}\end{addmargin}

\begin{enumerate}[
      leftmargin=0pt, labelsep=0.25em,
      label=\textsf{\textbf{\alph*)}}
]
\begin{multicols}{2}
\raggedcolumns

   \item “No ducks are willing to waltz.”
   \\ \Pm{¬\exists x (\: P(x) \wedge S(x) \:)}

   \item “No officers ever decline to waltz.”
   \\ \Pm{¬\exists x (\: R(x) \wedge ¬S(x) \:)}

   \item “All my poultry are ducks.”
   \\ \Pm{\forall x (\: Q(x) \rightarrow P(x) \:)}

   \item “My poultry are not officers.”
   \\ \Pm{\forall x (\: Q(x) \rightarrow ¬R(x) \:)}

\end{multicols}

   \item “Does \textit{d} follow from \textit{a}, \textit{b}, and \textit{c}? If not, is there a
   correct conclusion?”

   Yes, \textit{d} follows from \textit{a}, \textit{b}, and \textit{c}.

\end{enumerate}

% ==== ==== ====
\section*{1.5 | № 6}
\begin{addmargin}[2.5em]{2.5em}{\sffamily
      “Let \Pm{C(x, y)} mean that student \Pv{x} is enrolled in class \Pv{y}, where the domain for
      \Pv{x} consists of all students in your school and the domain for \Pv{y} consists of all
      classes being given at your school. Express each of these statements by a simple English
      sentence.”
}\end{addmargin}

\begin{enumerate}[
      leftmargin=0pt, labelsep=0.25em,
      label=\textsf{\textbf{\alph*)}}
]
\begin{multicols}{2}
\raggedcolumns



\end{multicols}
\end{enumerate}

\end{document}
