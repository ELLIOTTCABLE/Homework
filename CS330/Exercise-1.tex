\synctex=1
\documentclass[
   paper=a4,
   fontsize=11pt,
   parskip=no,
   fleqn             % Align math left? o_O
]{scrartcl}

\usepackage[svgnames]{xcolor}
\usepackage{graphics}
\usepackage[
   textheight=700px,
   head=18.27997pt
]{geometry}
\usepackage{etoolbox}

\usepackage{scrlayer-scrpage}
\usepackage{enumerate}
\usepackage{multicol}
   \setlength\columnsep{20pt}
   \setlength{\columnseprule}{0.5pt}
   \def\columnseprulecolor{\color{gray}}

\usepackage{amsmath,amsfonts,amsthm}
\usepackage{mathtools}
\usepackage[math-style=TeX]{unicode-math}
   \setmathfont{xits-math.otf}
%  \setmathfont[Scale=MatchLowercase,math-style=upright,vargreek-shape=unicode]{Neo Euler}
%  \setmathfont[range=\mathbfsfit/{greek,Greek,latin,Latin}]{EB Garamond}

\usepackage{fontspec}
   %\defaultfontfeatures{}
   \setmainfont{EB Garamond}[Numbers={OldStyle,Proportional}]
   \setsansfont{Linux Biolinum O}
   \setmonofont{Fantasque Sans Mono}
\setkomafont{disposition}{\normalfont}
\usepackage{bm}         % Bold-math handling

\usepackage{url}        % Pretty, line-breaking URLs                                            \url
\usepackage{currfile}   % Get the current filename (for the footer)                    \currfilename
\usepackage{lastpage}   % A reference to the page-number of the last page         \pageref{LastPage}

\usepackage{blindtext}

%%% === Layout ===
%\pagestyle{empty}
\automark[subsection]{section}
\pagestyle{scrheadings}

%%% Title
\setkomafont{author}{\scshape}

\title{Assignment 1}
\subtitle{CS 330}
\author{elliottcable}

%%% Header / footer
\setkomafont{pagehead}{\normalfont\sffamily\upshape}
\setkomafont{pagefoot}{\normalfont\sffamily\upshape}
\setkomafont{pagenumber}{\normalfont\sffamily\upshape}

\renewcommand\pagemark{{\usekomafont{pagenumber}%
   \thepage\nobreakspace of\nobreakspace\pageref{LastPage}%
}}

\lohead{\bfseries\headmark}    %% Top left on odd pages (and single-sided)
\rohead{\bfseries\pagemark}    %% Top right on odd pages (and single-sided)
\lofoot{\ttfamily\currfilename}
\rofoot{\today}
\chead{}                       %% Top center
\cfoot{}                       %% Bottom center

%%% Headings
%\usepackage{sectsty}
%   \sectionfont{%
%      \usefont{OT1}{phv}{b}{n}%
%      \sectionrule{0pt}{0pt}{-5pt}{3pt} }

%%% Patch the title to remove the date
\makeatletter
\patchcmd{\@maketitle}% <cmd>
   {{\usekomafont{date}{\@date \par}}%
      \vskip \z@ \@plus 1em}% <search>
   {}% <replace>
   {}{}% <success><failure>
\makeatother

%%% === Content ===
\begin{document}
\maketitle

\section*{№ 8}

\begin{multicols}{2}
\begin{enumerate}[a)]
   \item
      Taking \textit{“Kwame will take a job in industry”} as $\bm{\mathsf{i}}$, and \textit{“Kwame
      will go to graduate school”} as $\bm{\mathsf{g}}$, then …

      \begin{align*}
         & \mathsf{i \vee g}                                         \\\equiv\;
         & \mathsf{¬(i \vee g) \equiv ¬i \wedge ¬g}
      \end{align*}

      Thus, we derive: \textit{“Kwame will not take a job in industry and Kwame will not go to
      graduate school,”} or more simply, \textit{“Kwame will not take a job in industry, nor go to
      graduate school.”}

   \item
      Taking \textit{“Yoshiko knows Java”} as $\bm{\mathsf{j}}$, and \textit{“Yoshiko knows
      calculus”} as $\bm{\mathsf{c}}$, then …

      \begin{align*}
         & \mathsf{j \wedge c}                                       \\\equiv\;
         & \mathsf{¬(j \wedge c) \equiv ¬j \vee ¬c}
      \end{align*}

      Thus, we derive: \textit{“Yoshiko does not know Java or Yoshiko does not know calculus”} or
      more simply, \textit{“Yoshiko does not know at least one of Java or calculus.”}

  %\columnbreak

   \item
      Taking \textit{“James is young”} as $\bm{\mathsf{y}}$, and \textit{“James is strong”} as
      $\bm{\mathsf{s}}$, then …

      \begin{align*}
         & \mathsf{y \wedge s}                                       \\\equiv\;
         & \mathsf{¬(y \wedge s) \equiv ¬y \vee ¬s}
      \end{align*}

      Thus, we derive: \textit{“James is not young or James is not strong”} or more simply,
      \textit{“James is old and/or weak.”}

   \item
      Taking \textit{“Rita will move to Oregon”} as $\bm{\mathsf{o}}$, and \textit{“Rita will move
      to Washington”} as $\bm{\mathsf{w}}$, then …

      \begin{align*}
         & \mathsf{o \vee w}                                         \\\equiv\;
         & \mathsf{¬(o \vee w) \equiv ¬o \wedge ¬w}
      \end{align*}

      Thus, we derive: \textit{“Rita will not move to Oregon and Rita will not move to Washington”}
      or more simply, \textit{“Rita will not move to either Oregon or Washington.”}
\end{enumerate}
\end{multicols}

\end{document}
