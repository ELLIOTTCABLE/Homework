\documentclass[man]{apa}

\usepackage{polyglossia}
\setdefaultlanguage{english}
\setotherlanguage{french}

\usepackage[autostyle,autopunct,csdisplay,french=guillemets]{csquotes}
\MakeAutoQuote{“}{”}
\MakeAutoQuote*{‘}{’}
\MakeForeignQuote{french}{«}{»}

\usepackage{ellipsis}

\title{Response to Schapiro and Chiaromonte on Proudhon}
\author{Elliott Cable}
\affiliation{Department of Computer Science}
%\ifapamodeman{%
%\note{\begin{flushleft}
%  William Revelle\\
%    Department of Psychology\\
%  Northwestern University\\
% Evanston, Illinois\\
%	60201\\
%	e-mail: revelle@northwestern.edu\\
%
%   \end{flushleft}}}
%{%else, i.e., in jou and doc mode
%%\note{Draft of \today}
%}

\shorttitle{Response to Schapiro \& Chiaromonte}
\rightheader{Response to Schapiro \& Chiaromonte}
\leftheader{Elliott Cable}

\acknowledgements{The version I e-mailed you got generated in the “journal” format, this one is formatted to the APA double-spaced specifications from the syllabus. (Ignore the extra first page, I have no idea what I'm doing with this typesetting program.)}

\begin{document}
\EnableQuotes
\maketitle
I have to admit some consternation in writing on a historical period I know embarrassingly little about. In reading Messrs. Schapiro and Chiaromonte's writings, I found myself spending more time researching what are, I assume, elementary and well-known — or at least so it seems — historical events. I can only hope this brings to me some value, perhaps that of a newcomer's oft-unusual viewpoint, instead of only serving to hinder my understanding of their arguments. Nevertheless, herein, I'll attempt to analyze their notions and estimations.

Schapiro's argument, perhaps, can be summed to the following: the famed ‘father of anarchy’, Pierre-Joseph Proudhon, was in fact no anarchist at all, but rather a self-inconsistent and demagogic supporter of \emph{autarchy}, pursuant to social change. To begin with, Schapiro implies that Proudhon primarily achieved an audience in the first place thanks to the nature of the times in which he wrote (notably including a desperately-underserved market of small, middle-class businessmen — Marx's ‘petite bourgeoisie’), and to the passion of his writings. He proceeds to cast doubt on many popularization of Proudhon's work, including the famous «La propriété, c'est le vol!»: \blockquote{It is only by reading Proudhon carefully — and fully — that it is possible to understand what he meant by “property” and why he regarded it as “theft”. A false impression of Proudhon's views on this, as well as on the other matters, is derived from such dicta.}

Throughout the work, Schapiro counter-asserts many other viewpoints that he feels are over-represented in both contemporary and later analyses: that Proudhon was a revolutionary (at least in the traditional, violent sense); that Proudhon was of, or for, the masses, as represented by the working-class; even that Proudhon was against statist violence. Collectively, it's clear Schapiro considers Proudhon to have failed to make his own arguments — that his arguments are only clear through the lens of time \ldots\ which segues into his primary argument, that those arguments amount to analogues of ‘modern’ (to him) fascism.

That last being, of course, what Chiaromonte most wishes to contradict: Schapiro's titular claim that Proudhon held accidentally-anachronistic views that amount to fascism (a topic, clearly, near-and-dear to both of these authors, these works having been written around the events of the 1940s, but to Chiaromonte in particular, having fled Mussolini in 1934, and enlisted against Franco.) He goes on to some declamatory effort to dispute this assertion: he effectively describes Schapiro's arguments as “inexcusably devious” slander against Proudhon, finally insisting that the Frenchman be lauded for “originality as a thinker” — for his unconventional and wide-ranging analysis, even when such depth and intellectual honesty led to ambiguity and disclarity.

Chiaromonte expends many words on contradicting Mr. Schapiro on the minutiæ of his arguments, some more convincingly than others: for instance, he rather-handily puts down Schapiro's claims about Proudhon's views on violence by demonstrating the Socratic methodology thereof; his demonstration of Schapiro's contextual mis-understanding of Proudhon's commentary on the Black's struggle in the American south is similarly convincing. He does not, however, manage to effectively \emph{demonstrate} others; for instance, his claims that Proudhon's views are presented with “perfect clarity,” nor the statement that it is “preposterous” for Proudhon to be seen as a supporter of dictatorship.

In some part due to this stylistic inconsistency, I find Chiaromonte's motivations altogether opaque — in some ways, he seems to share Schapiro's relatively dry, academic interest in ‘setting the historical record straight’; but at other times, he seems more emotionally tied to the eventual perception of Proudhon. Regardless, I found Schapiro's \emph{arguments} much more convincing than Chiaromonte's. Perhaps this has something to do with a taste for his academic, unimpassioned style, or simply from the fecundity of his primary-source references. That does not, of course, mean I agree with him — when it's said and done, I know very little of the man being discussed, and I hesitate to form an opinion based on either of these treatises without reading further analyses, or even some of aforementioned primary sources.

At the completion of my analysis, I'm left with what seems a somewhat well-rounded view of Proudhon: although the majority of Schapiro's arguments ring true through Chiaromonte's rebuttal, the latter has perhaps ‘sanded’ some of the edges thereof. If nothing else, as Mr. Chiaromonte similarly observes at the end of her piece, Mr. Proudhon seems to incite considerable, even ardent, debate — despite (or perhaps in correction of?) my initial consternation, I've come away from these pieces with an uncharacteristic and crisp desire to read more of this \emph{philosophe} and his “uncomfortable thought.”

\end{document}
